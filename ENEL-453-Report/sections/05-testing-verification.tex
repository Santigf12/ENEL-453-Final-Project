\section{Testing and Verification}

For the testing and validation of the FPGA design and comparator circuit. Mostly in terms of creating the final breadboard circuit we tested the connections
by using the oscilloscope and the multimeter by analyzing graph patterns, reading voltage signals and measuring current.

In contrast, the way we ended up testing the RTL code against, we used a more crude organic approach, instead of using test bench files or something similar 
we used real time testing by making changes in the RTL code and synthesizing right way this was possible to do to the great performance of Vivado in a Fedora Linux machine, 
as well as, performance settings, this greatly reduced the generation to less than 10 minutes.

This type of testing was only possible due to the specific circumstances of the design, this approach should be reevaluated in future work closer to actual industry practices.